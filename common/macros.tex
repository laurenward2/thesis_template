% THE TITLE OF THE THESIS
\def\myname{Your Name Here}
\def\myadvisor{Your Advisor's Name}
\def\mytitle{A Template Dissertation for Students of Physics and Astronomy}
\def\myyear{2022}

% citation style

% default: cite with (Name, year)
\renewcommand{\cite}{\citep}

% common abbreviations
\newcommand{\eg}{{\it e.g.}\xspace}
\newcommand{\ie}{{\it i.e.}\xspace}
\newcommand{\etc}{{\it etc.}\xspace}
\newcommand{\etal}{\emph{et~al}\mbox{.}\xspace}

\newcommand{\vs}{{vs\mbox{.}}\xspace}

\newcommand{\defeq}[0]{\triangleq}
\renewcommand{\mod}{\operatorname{mod}}


% algorithm names
\newcommand{\kwfont}[1]{\textsf{#1}\xspace} %\small
% variable name
\newcommand{\var}[1]{\ensuremath{{\fun{#1}}}\xspace} %\small

%http://hstuart.dk/2007/08/03/programming-latex-%E2%80%94-writing-commands/
%\newcommand{\mkkw}[2]{
%	\newcommand{#1}[0]{\kwfont{#2}}
%}


\newcommand{\MESA}{{\tt MESA}\xspace}
\newcommand{\WDEC}{{\tt WDEC}\xspace}
\newcommand{\BV}{{Brunt-V\"ais\"al\"a}\xspace}

%basic math commands
\newcommand{\into}{\rightarrow}
\newcommand{\abs}[1]{\left\vert #1 \right\vert}
%derivaties
\newcommand{\del}[2]{\frac{\partial#1}{\partial #2}}
\newcommand{\dif}[2]{\frac{d#1}{d#2}}
%shortcuts
\newcommand{\xhat}{x}
\newcommand{\rhat}{\hat{r}}
\newcommand{\rbar}{\bar{r}}
\newcommand{\Mstar}{{M_\star}}
\newcommand{\Rstar}{{R_\star}}
\newcommand{\Lstar}{{L_\star}}
\newcommand{\Msolar}{{M_\odot}}
\newcommand{\Rsolar}{{R_\odot}}
\newcommand{\Lsolar}{{L_\odot}}
\newcommand{\Rearth}{{R_{\oplus}}}
\newcommand{\ad}{\mathrm{ad}}
\newcommand{\rad}{\mathrm{rad}}
\newcommand{\Teff}{{T_{\mathrm{eff}}}}



\renewcommand{\ln}{\log}

\newcommand{\pn}[1]{{\sc#1pn}\xspace}  

% src code
\newcommand{\src}[1]{\textsf{\small #1}\xspace}
\newcommand{\tcode}[1]{\text{\tt#1}}

%vector stuff
\newcommand{\bvec}[1]{{\vec{#1}\,}} % a vector with space to breathe!  use with superscripts
\newcommand{\bracket}[2]{\left\langle#1,#2\right\rangle} % an inner product
\newcommand{\m}[1]{\mathbf{#1}}
\newcommand{\mxi}{\boldsymbol{\xi}}
\newcommand{\dotprod}{\boldsymbol{\cdot}} % the typical scalar product
\newcommand{\innerprod}[2]{\left\langle#1\right\vert\left.\vphantom{#1}#2\right\rangle}
\newcommand{\innerprodN}[2]{\left(#1\right\vert\left.\vphantom{#1}#2\right)}


% references
\newcommand{\chref}[1]{Chapter~\ref{ch:#1}\xspace}
\newcommand{\chrefs}[2]{Chapters~\ref{ch:#1} and~\ref{ch:#2}\xspace}
\newcommand{\secref}[1]{Section~\ref{sec:#1}\xspace}
\newcommand{\subsecref}[1]{Section~\ref{subsec:#1}\xspace}
\newcommand{\figref}[1]{Figure~\ref{fig:#1}\xspace}
\newcommand{\figrefi}[2]{Figure~\ref{fig:#1}(#2)\xspace}
\newcommand{\tabref}[1]{Table~\ref{tab:#1}\xspace}
\newcommand{\equref}[1]{(\ref{eq:#1})\xspace}
\newcommand{\equsref}[1]{(\ref{eq:#1})\xspace}
\newcommand{\equrefs}[2]{(\ref{eq:#1},\ref{eq:#2})\xspace}
\newcommand{\pref}[1]{page~\pageref{p:#1}\xspace}
% citations

% parameters

\newcommand{\pacc}[0]{\var{pacc}}
\newcommand{\wratio}[0]{\var{wratio}}

% special footnotes

% from http://help-csli.stanford.edu/tex/latex-footnotes.shtml
\long\def\symbolfootnote[#1]#2{\begingroup%
\def\thefootnote{\fnsymbol{footnote}}\footnote[#1]{#2}\endgroup}

%% Flush words right at end of paragraph.
%% From: http://tex.stackexchange.com/questions/16330/hfill-after-linebreak
\newcommand\rightparend[1]{{%
      \unskip\nobreak\hfil\penalty50
      \hskip2em\hbox{}\nobreak\hfil\textbf{#1}%
      \parfillskip=0pt \finalhyphendemerits=0 \par}}


\defcitealias{FullLai11}{FL11}
\defcitealias{PresTeuk77}{PT77}
\defcitealias{ChriMull94}{JCD-DJM}
\defcitealias{PoisWill14}{P\&W}
\defcitealias{MisnThorWhee73}{MTW}

